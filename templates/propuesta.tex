\documentclass[
  12pt,
  %twocolumn
]{article}

\usepackage{graphicx}
\graphicspath{{figuras/}}

\usepackage{enumerate}

\usepackage{ifpdf}

\oddsidemargin=-0.25in
\evensidemargin=0pt
\textwidth=6.5in
\topmargin=-0.25in
\headheight=0pt
\headsep=0pt
\textheight=9.5in
\parindent=0pt                         % sin sangria
\parskip=9pt

%\addtolength{\hoffset}{-0.125in}       % correccion para mi impresora

\pagestyle{empty}

\newcommand{\pts}[1]{ {\it (#1 pts)}}
\newcommand{\cpts}[1]{ {\it (#1 pts/u)}}

\newcommand{\ii}{\hat{\imath}}
\newcommand{\jj}{\hat{\jmath}}
\newcommand{\kk}{\hat{k}}
\renewcommand{\sin}{\, \mathrm{sen}\,}
\newcommand{\sen}{\, \mathrm{sen}}
%\renewcommand{\lim}{\, \textrm{l\'im}\,}
\newcommand{\negrita}[1]{\mbox{\boldmath $ #1 $}}
\newcommand{\vect}[1]{\negrita{\vec{#1}}}
\newcommand{\espacio}{\vskip 5ex}

\newcommand{\pro}{promedio}
\newcommand{\spro}[2]{#1_{#2 \textrm{\scriptsize prom}}}
\newcommand{\sub}[2]{#1_{\textrm{\scriptsize #2}}}
\newcommand{\MUR}{a velocidad cons\-tan\-te}
\newcommand{\MUA}{con ace\-le\-ra\-ci\'on constante}
\newcommand{\CMt}{CM}
\newcommand{\CM}{\textrm{\tiny CM}}
\newcommand{\tot}{\textrm{\tiny tot}}
\newcommand{\obj}{\textrm{\tiny obj}}

\usepackage{amsmath}
\newcommand{\unit}[1]{\:\mathrm{#1}}
\begin{document}

\section{Trabajo y energía}
En una barco se tiene un motor especial para subir el ancla de manera que 
el ancla sube a velocidad constante. El ancla, de masa $m = 1250\:\mathrm{kg}$, está
amarrada por una cuerda gruesa de una densidad de $\lambda = 2.33\:\mathrm{kg/m}$,
que se encuentra de forma vertical y luego pasa por una polea de masa
despreciable, para  luego ser arrollada por el motor. 
El motor levanta el ancla que desde el fondo del mar hasta una altura de 
$h = 56.0\:\mathrm{m}$, quedando solo $\ell = 2.00\:\mathrm{m}$ de la cuerda por fuera
de la polea colgando.
\begin{enumerate}[a)]
  \item Plantee el diagrama de fuerzas para la cuerda colgando y las aplicación 
    de las leyes de newton según corresponde \pts{9}.
  \item Plantee la fuerza necesaria para subir con velocidad contante a 
    la cuerda con el ancla en función del largo $y$ que de cuerda que 
    queda colgando \pts{4}.
  \item Calcule el trabajo realizado por el motor al subir el ancla a velocidad
    constante \pts{12}.
\end{enumerate}
\begin{center}
  \includegraphics[width = 0.4\textwidth]{barco.png}
\end{center}
  \paragraph{Solución:}
\begin{enumerate}[a)]
  \item  El diagrama de fuerzas tiene la fuerza del motor, el peso de la cuerda 
    que cuelga y el peso del ancla.
   \[
     \sum F_y = F - \lambda y g - m g = 0
   .\] 
  \item A partir de las sumas de fuerzas.
\[
  F = g\left( m + \lambda y \right) 
.\] 
donde $y$ representa la cantidad de cuerda colgando.
  \item Para calcular el trabajo realizado necesitamos los datos:
  \[\begin{array}{l}
    \lambda = 2.33\unit{kg/m} \\
    m = 1250\unit{kg}
    y_i = h = 56.0\unit{m}\\
    y_f = \ell = 2.00\unit{m}\\
  \end{array}\]
  Note que es una fuerza variable que se integra desde que la cuerda está completamente
  estirada hasta que se ha recogido la cantidad indicada
  \[
    W = \int_{y_i}^{y_f} F(+\jj) \cdot \;\mathrm{d}y(-\jj)
  \] 
  \[
    W = -\int_h^{\ell} g\left( m + \lambda y \right) \;\mathrm{d}y
  .\] 
  \[
    W = -g\left. \left( m y + \frac{\lambda y^2}{2} \right)  \right|_h^{\ell}
  .\] 
  \[
    W = g\left( m (h - \ell ) + \frac{\lambda}{2}(h^2-\ell^2) \right) 
  .\] 
  \[
    W = 9.81 \left(  1250(56.0-2.00) + \frac{2.33}{2}(56.0^2-2.00^2) \right) 
    = 6.98\times 10^{6}\unit{J} 
  .\] 

  \paragraph{Distribución de puntos}
  \renewcommand{\labelenumiii}{\theenumi\theenumii-\arabic{enumiii})}
  \begin{enumerate}[a)]
    \item .
    \begin{enumerate}
      \item Identifica la fuerza del motor y su dirección - \pts{2}.
      \item Identifica el peso del ancla y su dirección - \pts{2}.
      \item Identifica el peso de la cuerda en función del lago que cuelga y su dirección
        - \pts{2}.
      \item Plantea la suma de fuerzas de forma correcta - \pts{3}.
    \end{enumerate}
    \item .
    \begin{enumerate}
      \item Despeja correctamente la fuerza del motor - \pts{2}.
      \item Indica correcta cuál una expresión lineal para la fuerza del motor - \pts{2}.
    \end{enumerate}
    \item .
    \begin{enumerate}
    \item Identifica que la fuerza va en dirección contraria de la dirección en
      que cuelga la cuerda - \pts{2}.
    \item Plantea el trabajo como la integral de la fuerza en función del desplazamiento
      realizado por la cuerda que cuelga - \pts{2}.
    \item Identifica que el trabajo es el resultado del producto punto entre la fuerza
      y el desplazamiento - \pts{2}
    \item Resuelve la integral de forma adecuada - \pts{4}.
    \item Resultado numérico del trabajo con la unidad correspondiente - \pts{2}
    \end{enumerate}
  \end{enumerate}
\end{enumerate}

\newpage
\section{Trabajo y energía}
En una montaña rusa que tiene un bucle, donde el carro, de masa $m = 2300\unit{kg}$,
pasas por la parte interior del bucle con una rapidez de $v_1 = 9.91 \unit{m/s}$.
Considere que el bucle se encuentra a una altura de $H = 25.0 \unit{m}$ sobre el suelo.

\begin{enumerate}[a)]
  \item Considerando que en punto más alto antes del bucle se encuentra a
    $h_i=20.0\unit{m}$, calcule la rapidez que debe tener en este punto \pts{12.5}.
  \item Si en el punto más bajo antes del bucle se tiene una rapidez de
    $v_2 = 19.8\unit{m /s}$, calcule la altura de este punto \pts{12.5}.
\end{enumerate}

\begin{center}
  \includegraphics[width = 0.7\textwidth]{MontanaRusa.png}
\end{center}

\paragraph{Solución:}
\begin{enumerate}[a)]
  \item .
  Datos:
  \[\begin{array}{l}
    v_1 = 9.91\unit{m/s}\\
    h_1 = H = 25.0 \unit{m}\\
    v_3 = ? \\
    h_3 = 20.0 \unit{m}
  \end{array}\]
\[
 \Delta K + \Delta U = 0
.\] 
\[
  \frac{mv_3^2}{2} - \frac{mv_1^2}{2} + mgh_3 - mgh_1 = 0
.\] 
\[
  v_2 = \sqrt{v_1^2 + 2g\left( h_1 - h_2 \right) }
  = \sqrt{9.91^2 + 2 \cdot 9.81 (25.0 - 20.0)} = 14.0\unit{m/s}
.\] 
  \item .
  Datos:
  \[\begin{array}{l}
    v_1 = 9.91\unit{m/s}\\
    h_1 = H = 25.0 \unit{m}\\
    v_2 = 19.8\unit{m/s} \\
    h_2 = ?
  \end{array}\]
\[
 \Delta K + \Delta U = 0
.\] 
\[
  \frac{mv_2^2}{2} - \frac{mv_1^2}{2} + mgh_2 - mgh_1 = 0
.\] 
\[
  h_2 = h_1 + \frac{v_1^2 - v_2^2}{2g}
  = 25.0 + \frac{9.91^2 - 19.8^2}{2\cdot 9.81} = 11.6\unit{m}
.\] 

  \paragraph{Distribución de puntos}
  \renewcommand{\labelenumiv}{\theenumi\theenumiii-\arabic{enumiv})}
  \begin{enumerate}[a)]
      \item .
    \begin{enumerate}
      \item Identifica las variables importantes - \pts{3}.
      \item Plantea la conservación de la energía - \pts{3}.
      \item Desarrollo matemático adecuado - \pts{4.5}.
      \item Resultado numérico final - \pts{2}.
    \end{enumerate}
      \item .
    \begin{enumerate}
      \item Identifica las variables importantes - \pts{3}.
      \item Plantea la conservación de la energía - \pts{3}.
      \item Desarrollo matemático adecuado - \pts{4.5}.
      \item Resultado numérico final - \pts{2}.
    \end{enumerate}
  \end{enumerate}
\end{enumerate}

\newpage
\setcounter{section}{0}
\section{Rotación Cuerpo Rígido}
Se tiene una cuerda arrollada a un cilindro de masa
$M = 10.0 \unit{kg}$ y radio $R = 25.0 \unit{cm}$.
El cilindro tiene un mecanismo que le permite girar
libremente sin fricción. Del extremo de la cuerda
libre se cuelga una masa  $m = 5.00 \unit{kg}$.
\begin{enumerate}[a)]
  \item Dibuje un diagrama de fuerzas tanto para el
    cilindro como para la masa colgante y la respectiva 
    ley de Newton para cada caso \pts{8}.
  \item Calcule la aceleración angular del cilindro \pts{7}.
  \item Suponga que sustituye la masa colgante por una
    fuerza del mismo valor, calcule la aceleración 
    angular del cilindro en este nuevo escenario \pts{5}.
  \item Calcule el momento de inercia del cilindro
    respecto al eje transversal al punto del cual cuelga
    la cuerda \pts{5}.
\end{enumerate}
\begin{center}
  \includegraphics[width=0.6\textwidth]{Cilindro.png}
\end{center}
\paragraph{Solución:}
\begin{enumerate}[a)]
  \item Cada diagrama da pie a las siguientes sumas de fuerzas y momentos de torsión :
    \[
      R T = I \alpha
    .\] 
    \[
      T - mg = -ma
    .\] 
  \item Datos:
  \[\begin{array}{l}
    M = 10.0 \unit{kg} \\
    R = 0.250 \unit{m}\\
    I = \frac{MR^2}{2} = 2.50 \unit{kg\: m^2} \\
    m = 5.00 \unit{kg}
  \end{array}\]
Usando las sumas de fuerzas y de momento de torsión
tenemos que
\[
  T = m\left( g - R\alpha \right) 
.\] 
\[
  Rm(g - R\alpha) = \frac{MR^2}{2}\alpha
.\] 
\[
  \alpha = \frac{2 mg}{R (M + 2m)}
= \frac{2 \cdot 5.00 \cdot 9.82 }{0.250 (10.0 + 2\cdot 5.00)}
= 19.6 \unit{rad/s}
.\] 
\item Datos:
  \[\begin{array}{l}
    M = 10.0 \unit{kg} \\
    R = 0.250 \unit{m}\\
    I = \frac{MR^2}{2} = 2.50 \unit{kg\: m^2} \\
    F = mg = 5.00 \cdot 9.81 = 49.0 \unit{N}
  \end{array}\]
  En este caso la suma de la masa colgante cambia, debido
  a que ya no hay masa:
  \[
    T- F = 0 \implies T = F
  .\] 
  \[
    R F = \frac{MR^2}{2}\alpha
  .\] 
  \[
    \alpha = \frac{2mg}{RM}
    = \frac{2\cdot 5.00 \cdot 9.81}{0.250 \cdot 10.0}
    = 39.2 \unit{rad /s}
  .\] 
  \item 
Datos:
\[\begin{array}{l}
    M = 10.0 \unit{kg} \\
    R = 0.250 \unit{m}\\
    I = \frac{MR^2}{2} = 2.50 \unit{kg\: m^2} \\
    d = R = 0.250\unit{m}
\end{array}\]
\[
  I = I_{cm} + M d^2
.\] 
\[
  I = \frac{MR^2}{2} + M R^2
.\] 
\[
  I = \frac{3MR^2}{2}
  = \frac{3\cdot 10.0  \cdot 0.250^2}{2}
  = 0.937 \unit{kg\:m^2}
.\] 
  \paragraph{Distribución de puntos}
  \renewcommand{\labelenumiv}{\theenumi\theenumiii-\arabic{enumiv})}
  \begin{enumerate}[a)]
      \item .
    \begin{enumerate}
    \item En el cilindro plantea la tensión de forma 
      tangencial y el radio como brazo de palanca - \pts{2}.
    \item En la masa colgante plantea el peso y la tensión
      como opuestos en la vertical - \pts{2}.
    \item Escribe de forma clara la suma de torsión para 
      el cilindro, indicando que la aceleración angular 
      comparte la dirección del momento de torsión - \pts{2}.
    \item Escribe de forma clara la suma de fuerzas para 
      la masa colgante, indicando que la aceleración
      comparte la dirección del peso (ya sea positiva
      o negativa según lo define el estudiante) - \pts{2}
    \end{enumerate}
      \item .
    \begin{enumerate}
      \item Identifica que la aceleración lineal de la
        masa colgante se relaciona con la aceleración
        angular del cilindro - \pts{1}.
      \item Despeje matemático correcto - \pts{5}.
      \item Resultado numérico correcto - \pts{1}.
    \end{enumerate}
      \item .
    \begin{enumerate}
      \item Reconoce que la suma de fuerzas asociada a la
        masa colgante cambia, de forma que el cálculo
        de la tensión cambia - \pts{2}.
      \item Desarrollo matemático - \pts{2}.
      \item Resultado numérico correcto - \pts{1}.
    \end{enumerate}
      \item .
    \begin{enumerate}
      \item Planteamiento del teorema de ejes paralelos
        - \pts{1}.
      \item Identifica que la distancia a correr el eje es
        igual al radio del cilindro - \pts{1}.
      \item Desarrollo matemático correcto - \pts{2}.
      \item Resultado numérico correcto - \pts{1}.
    \end{enumerate}
  \end{enumerate}
\end{enumerate}

\newpage
\section{Rotación Cuerpo Rígido}
Usted construye una rueda que consiste en una varilla delgada
doblada formando un anillo de radio $R = 30.0\unit{cm}$ y
masa $M = 45.0 \unit{g}$. Además le pone 5 varillas, cada
de masa $m = 15.0 \unit{g}$, que se unen en el centro y
cada una se une a un punto diferente en el anillo exterior.

\begin{enumerate}[a)]
  \item Calcule el momento de inercia de la rueda completa
    \pts{7}.
  \item Calcule el momento de inercia respecto a un eje 
    transversal que pasa por el punto de contacto entre 
    la rueda y el suelo \pts{7}.
  \item Calcule la aceleración que experimenta la 
    llanta si la pone a rodar cuesta abajo en una 
    calle con una inclinación de $\theta = 10.0^{\circ}$ 
    \pts{11}.
\end{enumerate}
\begin{center}
  \includegraphics[width=0.5\textwidth]{Rueda.png}
\end{center}

\paragraph{Solución:}
\begin{enumerate}[a)]
  \item El momento de inercia total es la suma del momento
    de inercia de cada parte. 
    Datos:
  \[\begin{array}{l}
    M = 0.450\unit{kg} \\
    R = 0.300\unit{m}\\
    m = 0.150\unit{kg} \\
    \ell = R
  \end{array}\]

Momento de inercia del anillo
\[
  I_a = MR^2
.\] 

Momento de inercia de una varilla
\[
  I_v = \frac{mR^2}{3}
.\] 
Momento de inercia total
\[
  I = I_a + 5 I_v
.\] 
\[
  I = MR^2 + 5 \frac{mR^2}{3}
  = 0.300^2\left( 0.450 + \frac{5\cdot 0.150}{5} \right) 
  = 6.30\times 10^{-2}\unit{kg\:m^2}
.\] 
  \item 
  Datos
  \[\begin{array}{l}
    I_{cm} = 0.937 \unit{kg\:m^2}\\
    d = R = 0.300\unit{m}\\
    m_{tot} = M + 5m
    = 0.450 + 5 \cdot 0.150
    = 1.20 \unit{kg}
  \end{array}\]
  Aplicando teorema de ejes paralelos
  \[
    I = I_{cm} + m_{tot} d^2
  .\] 
  \[
    I = MR^2 + \frac{5mR^2}{3} + MR^2 + mR^2
  .\] 
  \[
    I = R^2\left( 2 M + \frac{8m}{3} \right) 
    = 0.300^2 \left( 2 \cdot 0.450 + \frac{8\cdot 0.150}{3} \right) 
    = 0.117 \unit{kg\:m^2}
  .\] 
  \item Aplicando la segunda ley de Newton para la rotación
    al rededor del punto de contacto con la superficie:
    \[
      R m_{tot} g \sen(\theta) = I \alpha
    .\] 
    \[
      \alpha = \frac{Rm_{tot}g}{I} \sen(\theta)
      = \frac{0.300 \cdot 1.20 \cdot 9.81}{0.117} \sen(10^{\circ})
      = 5.24 \unit{rad/s}
    .\] 
  \item .
Datos:
\[\begin{array}{l}
  .
\end{array}\]

  \paragraph{Distribución de puntos}
  \renewcommand{\labelenumiv}{\theenumi\theenumiii-\arabic{enumiv})}
  \begin{enumerate}[a)]
      \item .
    \begin{enumerate}
      \item Identifica correctamente el momento de inercia para el anillo - \pts{1}.
      \item Identifica correctamente el momento de inercia para cada varilla - \pts{1}.
      \item Reconoce que debe sumar los momentos de inercia, de manera que
        se suma cinco veces el de la varilla - \pts{3}.
      \item Resultado numérico correcto con la unidades apropiadas - \pts{2}
    \end{enumerate}
      \item .
    \begin{enumerate}
    \item Identifica que la distancia que debe moverse el momento de inercia es
      el radio - \pts{1}.
    \item Identifica que se usa la masa total de la rueda - \pts{1}.
    \item Utiliza el teorema de ejes paralelos - \pts{1}.
    \item Despeje matemático correcto - \pts{2}.
    \item Resultado numérico correcto con unidades apropiadas - \pts{2}
    \end{enumerate}
      \item .
    \begin{enumerate}
      \item Calcula correctamente el momento de inercia del peso - \pts{4}.
      \item Aplica la segunda ley de Newton para la rotación - \pts{2}.
      \item Identifica que debe utilizar el momento de inercia respecto
        al borde - \pts{1}.
      \item Desarrollo matemático correcto - \pts{2}.
      \item Resultado numérico correcto con las unidades correctas - \pts{2}.
    \end{enumerate}
  \end{enumerate}
\end{enumerate}

\end{document}
